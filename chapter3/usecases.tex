\section{Use Cases} % (fold)
\label{sec:use_cases}

From the point of view of the user, there will be not one but two different interfaces.
Each one having particular restrictions and abstractions.
Therefore, it is useful to distinguish between the additional use cases for the desktop interface, and the separate use cases for the mobile interface.

\subsection{New Use Cases} % (fold)
\label{sub:new_use_cases}

From the previous requirements, there are new use cases that needs to be addressed, mostly from the second phase.
For starters, three important scenarios are related to the new content tab, to cover exactly the three use cases missing in Figure~\ref{fig:usecasesiptv}.

\begin{center}
  \begin{usecase}[Buy new content]
    \label{tab:usecasebuy}%
    \usecaseactor{System user}
    \usecasepre{The user must have at least one device online. Also, the sidebar must be opened having the active tab set to \emph{Content}.}
    \usecasepost{A new session must start on the default user device, or if that device is offline, on one of the online devices. The user must be notified with a popup and the session icon must appear in the \ida{PNAI} interface for that device.}
    \usecasemain{
      \begin{usecasepath}
        \item User clicks on the \emph{Buy} (\emph{Play} if the content is free) button of the wanted content.
        \item A popup window appears asking for confirmation.
        \item The user confirms the action clicking on the button.
        \item The popup window closes.
        \item A popup appears to notify the user that the action is in progress.
        \item The content starts playing on the default device, or another online device if it is disconnected.
        \item The popup disappears and a session icon is created inside of the destination device.
      \end{usecasepath}
    }
    \usecasealt{1}{
      \begin{usecasepath}[b]
        \setcounter{enumi}{2}
        \item The user cancels the action.
        \item The popup window closes and action is cancelled.
      \end{usecasepath}
    }
    \usecasealt{2}{
      \begin{usecasepath}[c]
        \setcounter{enumi}{6}
        \item There is an error with the server and the session does not initiate.
        \item The content of the popup changes to notify the user that there was an error with the server and the action could not be completed.
        After 5 seconds it disappears.
        \item Action is cancelled.
      \end{usecasepath}
    }
  \end{usecase}
\end{center}

\begin{center}
  \begin{usecase}[Buy new content onto a device]
    \label{tab:usecasebuydevice}%
    \usecaseactor{System user}
    \usecasepre{The user must have at least one device online. Also, the sidebar must be opened having the active tab set to \emph{Content}.}
    \usecasepost{A new session must start on that selected device. The user must be notified with a popup and the session icon must appear in the \ida{PNAI} interface for that device.}
    \usecasemain{
      \begin{usecasepath}
        \item User starts dragging the content icon.
        \item A copy of the content icon appears under the user's cursor, and follows the cursor until the user drops it.
        \item User drops the cloned content icon into another device that is online.
        \item A popup window appears asking for confirmation.
        \item The user confirms the action clicking on the button.
        \item The popup window closes.
        \item A popup appears to notify the user that the action is in progress.
        \item The content starts playing on that device.
        \item The popup disappears and a session icon is created inside of the destination device.
      \end{usecasepath}
    }
    \usecasealt{1}{
      \begin{usecasepath}[b]
        \setcounter{enumi}{1}
        \item User drops the content icon into a blank space.
        \item Action is cancelled.
      \end{usecasepath}
    }
    \usecasealt{2}{
      \begin{usecasepath}[c]
        \setcounter{enumi}{3}
        \item The user cancels the action.
        \item The popup window closes and action is cancelled.
      \end{usecasepath}
    }
    \usecasealt{3}{
      \begin{usecasepath}[c]
        \setcounter{enumi}{7}
        \item There is an error with the server and the session does not initiate.
        \item The content of the popup changes to notify the user that there was an error with the server and the action could not be completed.
        After 5 seconds it disappears.
        \item Action is cancelled.
      \end{usecasepath}
    }
  \end{usecase}
\end{center}

\begin{center}
  \begin{usecase}[Buy new content onto a buddy]
    \label{tab:usecasebuybuddy}%
    \usecaseactor{System user}
    \usecasepre{The user must have at least one buddy online. Also, the sidebar must be opened having the active tab set to \emph{Content}.}
    \usecasepost{A new session must start on the default (or online) device of that selected buddy. The user must be notified with a popup and the session icon must appear in the \ida{PNAI} interface for that buddy.}
    \usecasemain{
      \begin{usecasepath}
        \item User starts dragging the content icon.
        \item A copy of the content icon appears under the user's cursor, and follows the cursor until the user drops it.
        \item Without releasing the icon, the user moves the mouse to the buddies tab, and keeps it there for a second.
        \item The buddies tab gets selected, revealing the buddies.
        \item User drops the cloned content icon into a buddy that is online.
        \item A popup window appears asking for confirmation.
        \item The user confirms the action clicking on the button.
        \item The popup window closes.
        \item A popup appears to notify the user that the action is in progress.
        \item The content starts playing on the default (or online) device of that selected buddy.
        \item The popup disappears and a session icon is created right to the destination buddy's name.
      \end{usecasepath}
    }
    \usecasealt{1}{
      \begin{usecasepath}[b]
        \setcounter{enumi}{1}
        \item User drops the content icon into a blank space.
        \item Action is cancelled.
      \end{usecasepath}
    }
    \usecasealt{2}{
      \begin{usecasepath}[b]
        \setcounter{enumi}{4}
        \item User drops the content icon into a blank space.
        \item Action is cancelled.
      \end{usecasepath}
    }
    \usecasealt{3}{
      \begin{usecasepath}[c]
        \setcounter{enumi}{6}
        \item The user cancels the action.
        \item The popup window closes and action is cancelled.
      \end{usecasepath}
    }
    \usecasealt{4}{
      \begin{usecasepath}[c]
        \setcounter{enumi}{9}
        \item There is an error with the server and the session does not initiate.
        \item The content of the popup changes to notify the user that there was an error with the server and the action could not be completed.
        After 5 seconds it disappears.
        \item Action is cancelled.
      \end{usecasepath}
    }
  \end{usecase}
\end{center}

To handle the requirements from UR9 to UR12, a new sidebar needs to be implemented with extra functionality.
This collapsible sidebar will have two tabs to control the content of the sidebar, triggering some new use cases.

\begin{center}
  \begin{usecase}[Resize sidebar]
    \label{tab:usecaseresizesidebar}%
    \usecaseactor{System user}
    \usecasepre{The application is fully loaded.}
    \usecasepost{The sidebar must be opened at the exact point the user released the mouse, within certain bounds.}
    \usecasemain{
      \begin{usecasepath}
        \item User starts dragging the left border of the sidebar.
        \item The width of the sidebar grows or shrinks following the mouse, readapting the device list to that space in every step.
        \item When the user releases the mouse, the sidebar stays in that size.
      \end{usecasepath}
    }
    \usecasealt{1}{
      \begin{usecasepath}[b]
        \setcounter{enumi}{2}
        \item The mouse goes to the half left part of the page or further to the left.
        \item The sidebar stays filling only the 50\% right part of the page.
        \item When the user releases the mouse, the sidebar stays in that size (half of the page).
      \end{usecasepath}
    }
  \end{usecase}
\end{center}

\begin{center}
  \begin{usecase}[Collapse sidebar]
    \label{tab:usecasecollapsesidebar}%
    \usecaseactor{System user}
    \usecasepre{The sidebar must be opened.}
    \usecasepost{The sidebar must be collapsed.}
    \usecasemain{
      \begin{usecasepath}
        \item User clicks on the sidebar handler.
        \item The sidebar is collapsed and the devices are readapted in that new space.
      \end{usecasepath}
    }
    \usecasealt{1}{
      \begin{usecasepath}[b]
        \setcounter{enumi}{0}
        \item User clicks on the currently selected tab.
        \item The sidebar is collapsed and the devices are readapted in that new space.
      \end{usecasepath}
    }
    \usecasealt{2}{
      \begin{usecasepath}[c]
        \setcounter{enumi}{0}
        \item User starts dragging the left border of the sidebar until its gets to the right part of the page (exactly to the 15\% space in the right).
        \item The sidebar is collapsed and the devices are readapted in that new space.
        \item User releases the mouse.
      \end{usecasepath}
    }
  \end{usecase}
\end{center}

\begin{center}
  \begin{usecase}[Open sidebar]
    \label{tab:usecaseopensidebar}%
    \usecaseactor{System user}
    \usecasepre{The sidebar must be collapsed.}
    \usecasepost{The sidebar must be opened.}
    \usecasemain{
      \begin{usecasepath}
        \item User clicks on the sidebar handler.
        \item The sidebar is opened and the devices are readapted in that new space.
        \item The previous selected tab gets selected.
      \end{usecasepath}
    }
    \usecasealt{1}{
      \begin{usecasepath}[b]
        \setcounter{enumi}{0}
        \item User clicks on any tab.
        \item The sidebar is opened and the devices are readapted in that new space.
        \item That tab gets selected.
      \end{usecasepath}
    }
    \usecasealt{2}{
      \begin{usecasepath}[c]
        \setcounter{enumi}{0}
        \item User starts dragging the left border of the sidebar until its gets to the left part of the page (exactly the 85\% space in the left).
        \item The sidebar is collapsed and the devices are readapted in that new space.
        \item The previous selected tab gets selected.
        \item User releases the mouse.
      \end{usecasepath}
    }
  \end{usecase}
\end{center}

\begin{center}
  \begin{usecase}[Select tab]
    \label{tab:usecaseselecttab}%
    \usecaseactor{System user}
    \usecasepre{The sidebar must be either collapsed or opened with the active tab set to the other tab.}
    \usecasepost{The tab must be selected, i.e., the sidebar should show the content for that tab.}
    \usecasemain{
      \begin{usecasepath}
        \item User clicks on the tab.
        \item The tab is selected and its content is changed to the content for that tab.
      \end{usecasepath}
    }
    \usecasealt{1}{
      \begin{usecasepath}[b]
        \setcounter{enumi}{1}
        \item If the sidebar is collapsed, then the sidebar is opened.
        \item The tab is selected and its content is changed to the content for that tab.
      \end{usecasepath}
    }
  \end{usecase}
\end{center}

The initial phase requirements also need some use cases, although less critical than the others.
For example, the ability of moving and resizing devices (UR6) can be divided into two use cases.

\begin{center}
  \begin{usecase}[Move device]
    \label{tab:usecasemovedevice}%
    \usecaseactor{System user}
    \usecasepre{There is at least one registered device for that user.}
    \usecasepost{The device is moved to a new position.}
    \usecasemain{
      \begin{usecasepath}
        \item User moves the mouse over the device he wants to move.
        \item Handlers for moving and resizing that device fade in.
        \item User starts dragging the handler or the title of that device (this title is covering the top of the user).
        \item The device moves following the mouse.
        \item When the user releases the mouse, the device stays in that position.
      \end{usecasepath}
    }
    \usecasealt{1}{
      \begin{usecasepath}[b]
        \setcounter{enumi}{3}
        \item The mouse is moved out of bounds.
        \item The device stays inside of those limits, in the nearest position to the mouse.
        \item When the user releases the mouse, the device stays in that position.
      \end{usecasepath}
    }
  \end{usecase}
\end{center}

\begin{center}
  \begin{usecase}[Resize device]
    \label{tab:usecaseresizedevice}%
    \usecaseactor{System user}
    \usecasepre{There is at least one registered device for that user.}
    \usecasepost{The device is resized.}
    \usecasemain{
      \begin{usecasepath}
        \item User moves the mouse over the device he wants to resize.
        \item Handlers for moving and resizing that device fade in.
        \item User starts dragging the resize handler (at the bottom right part of the device).
        \item The device resizes following the mouse, keeping its relative ratio.
        These limits must ensure that the device icon is fully displayed.
        \item When the user releases the mouse, the device stays in that size.
      \end{usecasepath}
    }
    \usecasealt{1}{
      \begin{usecasepath}[b]
        \setcounter{enumi}{3}
        \item The mouse is moved out of bounds.
        \item The device stays inside of the fixed limits, in the nearest position to the mouse.
        These limits must ensure that the device icon is fully displayed and that there are enough space for the session icons to be drawn inside.
        \item When the user releases the mouse, the device stays in that size.
      \end{usecasepath}
    }
  \end{usecase}
\end{center}

% subsection new_use_cases (end)

\subsection{Mobile Use Cases} % (fold)
\label{sub:mobile_use_cases}

All of the previous use cases are defined considering the desktop interface.
However, as the third phase consists on developing another very different interface, more use cases have to be defined for that mobile version.
Some of the new use cases, like resizing devices or the sidebar, do not need to be implemented, since those components do not make sense in the mobile interface.

The \ida{PNAI} interface is only a frame inside of the \idx{ScaleNet} interface, but the mobile version is a full application outside of any frame.
So there are a few more things that need to be handle, like user identification actions.
The rest of the actions, equivalent to the desktop version, have to be changed to follow more appropriate interaction patterns.

\begin{center}
  \begin{usecase}[Login (mobile)]
    \label{tab:usecaseloginmobile}%
    \usecaseactor{System user}
    \usecasepre{The application is fully loaded.}
    \usecasepost{The user must be identified, and the interface with the rest of the actions must be shown.}
    \usecasemain{
      \begin{usecasepath}
        \item User enters its name and password, and taps on the \emph{Login} button.
        \item Credentials are checked.
        \item Credentials are correct, so the user interface loads with his devices and buddies.
      \end{usecasepath}
    }
    \usecasealt{1}{
      \begin{usecasepath}[b]
        \setcounter{enumi}{2}
        \item Credentials are incorrect, so the same form is shown with a message to inform that the credentials are incorrect.
      \end{usecasepath}
    }
  \end{usecase}
\end{center}

\begin{center}
  \begin{usecase}[Logout (mobile)]
    \label{tab:usecaselogoutmobile}%
    \usecaseactor{System user}
    \usecasepre{The application is fully loaded and it is not blocked, and the user is logged in.}
    \usecasepost{The session must end, showing the login form.}
    \usecasemain{
      \begin{usecasepath}
        \item User taps on the \emph{Logout} button.
        \item The session is deleted.
        \item The login form is shown.
      \end{usecasepath}
    }
  \end{usecase}
\end{center}

\begin{center}
  \begin{usecase}[Select tab (mobile)]
    \label{tab:usecaseselecttabmobile}%
    \usecaseactor{System user}
    \usecasepre{The application is fully loaded and it is not blocked, and the user is logged in.}
    \usecasepost{The tab must be selected, i.e., the main view should show the content for that tab.}
    \usecasemain{
      \begin{usecasepath}
        \item User taps on the tab.
        \item The tab is selected and the main view is changed to show the content related to that tab.
      \end{usecasepath}
    }
  \end{usecase}
\end{center}

\begin{center}
  \begin{usecase}[Stop a session of a device (mobile)]
    \label{tab:usecasestopdevicemobile}%
    \usecaseactor{System user}
    \usecasepre{A session is already running on a device, and it is showing in the \ida{PNAI} interface near that device's icon.}
    \usecasepost{Session must terminate, i.e., the content must stop playing. The users must be notified with a popup and the session icon must be deleted from the \ida{PNAI} interface.}
    \usecasemain{
      \begin{usecasepath}
        \item User taps on the session icon.
        \item A menu appears over the session icon.
        \item User taps the \emph{Stop} option.
        \item The popup menu disappears.
        \item A popup appears to notify the user that the action is in progress.
        \item The content stops playing.
        \item The popup disappears and the session icon is deleted from the view.
      \end{usecasepath}
    }
    \usecasealt{1}{
      \begin{usecasepath}[b]
        \setcounter{enumi}{2}
        \item User taps outside the menu.
        \item The popup menu disappears.
        \item Action is cancelled.
      \end{usecasepath}
    }
    \usecasealt{2}{
      \begin{usecasepath}[c]
        \setcounter{enumi}{5}
        \item There is an error with the server and the content keeps playing.
        \item The content of the popup changes to notify the user that there was an error with the server and the action could not be completed.
        After 5 seconds it disappears.
        \item Action is cancelled.
      \end{usecasepath}
    }
  \end{usecase}
\end{center}

\begin{center}
  \begin{usecase}[Stop a session of a buddy (mobile)]
    \label{tab:usecasestopbuddymobile}%
    \usecaseactor{System user}
    \usecasepre{A session owned by the user is running on a device, and it is showing in the \ida{PNAI} interface near that buddy's name.}
    \usecasepost{Session must terminate, i.e., the content must stop playing. The user must be notified with a popup and the session icon must be deleted from the \ida{PNAI} interface. The buddy is \emph{not} notified, the content stops without warning.}
    \usecasemain{
      \begin{usecasepath}
        \item User taps on the session icon.
        \item A menu appears over the session icon.
        \item User taps the \emph{Stop} option.
        \item The popup menu disappears.
        \item A popup appears to notify the user that the action is in progress.
        \item The content stops playing.
        \item The popup disappears and the session icon is deleted from the view.
      \end{usecasepath}
    }
    \usecasealt{1}{
      \begin{usecasepath}[b]
        \setcounter{enumi}{2}
        \item User taps outside the menu.
        \item The popup menu disappears.
        \item Action is cancelled.
      \end{usecasepath}
    }
    \usecasealt{2}{
      \begin{usecasepath}[c]
        \setcounter{enumi}{5}
        \item There is an error with the server and the content keeps playing.
        \item The content of the popup changes to notify the user that there was an error with the server and the action could not be completed.
        After 5 seconds it disappears.
        \item Action is cancelled.
      \end{usecasepath}
    }
  \end{usecase}
\end{center}

\begin{center}
  \begin{usecase}[Copy a session to a device (mobile)]
    \label{tab:usecasecopydevicemobile}%
    \usecaseactor{System user}
    \usecasepre{A session is already running on a device/buddy, and it is showing in the \ida{PNAI} interface near that device/buddy. Also, there is another device online.}
    \usecasepost{Session must be copied to that device, i.e., the content must be duplicated and played on that device. The user must be notified with a popup and the session icon must appear in the \ida{PNAI} interface for the second device.}
    \usecasemain{
      \begin{usecasepath}
        \item User taps on the session icon.
        \item A menu appears over the session icon.
        \item User taps the \emph{Duplicate} option.
        \item The popup menu disappears.
        \item The interface changes to notify the user that now he has to select the destination device.
        \item If the devices tab is not selected, user taps on the devices tab icon.
        \item User taps on the device he wants.
        \item The interface changes back to the normal state.
        \item A popup appears to notify the user that the action is in progress.
        \item The content starts playing in the other device.
        \item The popup disappears and the session icon is created near the destination device.
      \end{usecasepath}
    }
    \usecasealt{1}{
      \begin{usecasepath}[b]
        \setcounter{enumi}{2}
        \item User taps outside the menu.
        \item The popup menu disappears.
        \item Action is cancelled.
      \end{usecasepath}
    }
    \usecasealt{2}{
      \begin{usecasepath}[c]
        \setcounter{enumi}{5}
        \item User taps the \emph{Cancel} button, or in the add content tab.
        \item The interface changes back to the normal state.
        \item Action is cancelled.
      \end{usecasepath}
    }
    \usecasealt{3}{
      \begin{usecasepath}[d]
        \setcounter{enumi}{9}
        \item There is an error with the server and the content is not duplicated.
        \item The content of the popup changes to notify the user that there was an error with the server and the action could not be completed.
        After 5 seconds it disappears.
        \item Action is cancelled.
      \end{usecasepath}
    }
  \end{usecase}
\end{center}

\begin{center}
  \begin{usecase}[Copy a session to a buddy (mobile)]
    \label{tab:usecasecopybuddymobile}%
    \usecaseactor{System user}
    \usecasepre{A session is already running on a device/buddy, and it is showing in the \ida{PNAI} interface near that device/buddy. Also, there is another buddy online.}
    \usecasepost{Session must be copied to that buddy, i.e., the content must be duplicated and played on the buddy's default device. The user must be notified with a popup and the session icon must appear in the \ida{PNAI} interface near the name of that buddy. The buddy is \emph{not} notified, the content plays without warning.}
    \usecasemain{
      \begin{usecasepath}
        \item User taps on the session icon.
        \item A menu appears over the session icon.
        \item User taps the \emph{Duplicate} option.
        \item The popup menu disappears.
        \item The interface changes to notify the user that now he has to select the destination buddy.
        \item If the buddies tab is not selected, user taps on the buddies tab icon.
        \item User taps on the buddy he wants.
        \item The interface changes back to the normal state.
        \item A popup appears to notify the user that the action is in progress.
        \item The content starts playing in the buddy's device.
        \item The popup disappears and the session icon is created near the destination device.
      \end{usecasepath}
    }
    \usecasealt{1}{
      \begin{usecasepath}[b]
        \setcounter{enumi}{2}
        \item User taps outside the menu.
        \item The popup menu disappears.
        \item Action is cancelled.
      \end{usecasepath}
    }
    \usecasealt{2}{
      \begin{usecasepath}[c]
        \setcounter{enumi}{5}
        \item User taps the \emph{Cancel} button, or in the \emph{Add Content} tab.
        \item The interface changes back to the normal state.
        \item Action is cancelled.
      \end{usecasepath}
    }
    \usecasealt{3}{
      \begin{usecasepath}[d]
        \setcounter{enumi}{9}
        \item There is an error with the server and the content is not duplicated.
        \item The content of the popup changes to notify the user that there was an error with the server and the action could not be completed.
        After 5 seconds it disappears.
        \item Action is cancelled.
      \end{usecasepath}
    }
  \end{usecase}
\end{center}

\begin{center}
  \begin{usecase}[Transfer a session to a device (mobile)]
    \label{tab:usecasetransferdevicemobile}%
    \usecaseactor{System user}
    \usecasepre{A session is already running on a device/buddy, and it is showing in the \ida{PNAI} interface near that device/buddy. Also, there is another device online.}
    \usecasepost{Session must be transferred to that device, i.e., playback must be stopped at the source and started at the destination device. The user must be notified with a popup and the session icon must appear in the \ida{PNAI} interface for the second device.}
    \usecasemain{
      \begin{usecasepath}
        \item User taps on the session icon.
        \item A menu appears over the session icon.
        \item User taps the \emph{Hand over} option.
        \item The popup menu disappears.
        \item The interface changes to notify the user that now he has to select the destination device.
        \item If the devices tab is not selected, user taps on the devices tab icon.
        \item User taps on the device he wants.
        \item The interface changes back to the normal state.
        \item A popup appears to notify the user that the action is in progress.
        \item The content stops playing in that device and starts playing in the other device.
        \item The popup disappears, the session icon is deleted from the device and created near the destination device.
      \end{usecasepath}
    }
    \usecasealt{1}{
      \begin{usecasepath}[b]
        \setcounter{enumi}{2}
        \item User taps outside the menu.
        \item The popup menu disappears.
        \item Action is cancelled.
      \end{usecasepath}
    }
    \usecasealt{2}{
      \begin{usecasepath}[c]
        \setcounter{enumi}{5}
        \item User taps the \emph{Cancel} button, or in the add content tab.
        \item The interface changes back to the normal state.
        \item Action is cancelled.
      \end{usecasepath}
    }
    \usecasealt{3}{
      \begin{usecasepath}[d]
        \setcounter{enumi}{9}
        \item There is an error with the server and the content is not transferred.
        \item The content of the popup changes to notify the user that there was an error with the server and the action could not be completed.
        After 5 seconds it disappears.
        \item Action is cancelled.
      \end{usecasepath}
    }
  \end{usecase}
\end{center}

\begin{center}
  \begin{usecase}[Transfer a session to a buddy (mobile)]
    \label{tab:usecasetransferbuddymobile}%
    \usecaseactor{System user}
    \usecasepre{A session is already running on a device/buddy, and it is showing in the \ida{PNAI} interface near that device/buddy. Also, there is another buddy online.}
    \usecasepost{Session must be transferred to that buddy, i.e., playback must be stopped at the source and started at the buddy's default device. The user must be notified with a popup and the session icon must appear in the \ida{PNAI} interface near the name of that buddy. The buddy is \emph{not} notified, the content plays without warning.}
    \usecasemain{
      \begin{usecasepath}
        \item User taps on the session icon.
        \item A menu appears over the session icon.
        \item User taps the \emph{Hand over} option.
        \item The popup menu disappears.
        \item The interface changes to notify the user that now he has to select the destination buddy.
        \item If the buddies tab is not selected, user taps on the buddies tab icon.
        \item User taps on the buddy he wants.
        \item The interface changes back to the normal state.
        \item A popup appears to notify the user that the action is in progress.
        \item The content stops playing in that device and starts playing in the buddy's device.
        \item The popup disappears, the session icon is deleted from the device and created near the destination device.
      \end{usecasepath}
    }
    \usecasealt{1}{
      \begin{usecasepath}[b]
        \setcounter{enumi}{2}
        \item User taps outside the menu.
        \item The popup menu disappears.
        \item Action is cancelled.
      \end{usecasepath}
    }
    \usecasealt{2}{
      \begin{usecasepath}[c]
        \setcounter{enumi}{5}
        \item User taps the \emph{Cancel} button, or in the \emph{Add Content} tab.
        \item The interface changes back to the normal state.
        \item Action is cancelled.
      \end{usecasepath}
    }
    \usecasealt{3}{
      \begin{usecasepath}[d]
        \setcounter{enumi}{9}
        \item There is an error with the server and the content is not transferred.
        \item The content of the popup changes to notify the user that there was an error with the server and the action could not be completed.
        After 5 seconds it disappears.
        \item Action is cancelled.
      \end{usecasepath}
    }
  \end{usecase}
\end{center}

% subsection mobile_use_cases (end)

% section use_cases (end)
\section{APE Server\index{APE server} Installation and Configuration} % (fold)
\label{sec:apeconfig}

In this section the installation and configuration of the \ac{APE} server\index{APE server} are defined step by step.

\subsection{Install the Server} % (fold)
\label{sub:apeinstall}

The \ac{APE} download page \cite{ApeDownload} contains packages for different operating systems and architectures.
In this case, since the system is Debian-based we should use the DEB package.
Once the correct package is downloaded, it can be installed on the \idx{Application Server} by typing Listing~\vref{apeinstallation} from the same directory as the package is stored.

\begin{lstlisting}[label=apeinstallation,caption=APE installation command]
sudo dpkg -i ape-1.0.i386.deb
\end{lstlisting}

After that, the \ac{APE} server daemon (\idc{aped}) is automatically started with the default configuration \cite{ApeSetup}.
It can be checked by visiting the url \url{webportal.imusu.mobile.dtrd.de:6969}.

\nicesubsectionending

% subsection apeinstall (end)

\subsection{Configure BIND} % (fold)
\label{sub:bind}

The \ida{IMS} core is the machine that provides the \idas{DNS} service through \idas{BIND}, and that service needs to be configured to allow the \ac{APE} server\index{APE server} to use a lot of different dynamic subdomains like \verb|1.ape.webportal|, \verb|2.ape.webportal|, \verb|567.ape.webportal|, etc.

This is how the \ac{APE} server\index{APE server} works by default, and it appears that there is no way to configure the \ac{APE} server\index{APE server} for using only one domain \cite{ApeConfig}.

So, in the file \verb|/etc/bind/imusu.dnszone| located in the \ida{IMS} core
we have to look for the \emph{webportal} entry and change that section to look
like Listing~\vref{bindconf}.

\begin{lstlisting}[label=bindconf,caption=BIND configuration]
webportal               1D IN A         192.168.5.234
ape.webportal           1D IN A         192.168.5.234
*.ape.webportal         1D IN CNAME     ape.webportal
\end{lstlisting}

To apply the changes, we have to restart \idas{BIND} using the command in
Listing~\vref{bindrestart}.

\begin{lstlisting}[label=bindrestart,caption=BIND restart command]
sudo /etc/init.d/bind restart
\end{lstlisting}

% subsection bind (end)

% section apeconfig (end)

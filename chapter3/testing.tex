\section{Software Validation and Verification} % (fold)
\label{sec:software_validation_and_verification}

Due to the nature of the service applications running in the demonstration, none of them include unit tests or any other form of automated software testing.
Though these tests are a proved tool for developing reliable software, the scope of this project is limited to demonstrations purposes and the resulting application will not be used in the wild by many people.

Therefore it is preferred to speed up the development as much as possible rather than testing every little possibility so that no minor bugs are introduced.
For these reasons this application did not include those tests, and the validation process consisted on trying all major use cases in all devices and browsers.
For development and testing there were several devices:

\begin{itemize}
  \item The main development laptop, running Ubuntu. To ease the testing phase, two virtual machines were used in this laptop\footnote{Microsoft offers free virtual machines to test websites with all major versions of \ida{IE}: \url{http://www.microsoft.com/download/en/details.aspx?id=11575}}:
  \begin{itemize}
    \item Windows Vista with \ida{IE} 7.
    \item Windows Vista with \ida{IE} 8.
  \end{itemize}
  \item A laptop running Windows XP.
  \item A desktop computer running Windows Vista, with a TV acting as its monitor screen.
  \item A laptop running Windows Vista.
  \item An \idx{iPhone 3G} running \idx{iOS} 3.
\end{itemize}

Most of the development time was spent on the Ubuntu laptop, using \idx{Google Chrome} as the default browser because of the fantastic debugging tools it has built in.
On the other machines, several browsers were tested in a regularly fashion to ensure the application to keep a certain compatibility with them.
Table~\vref{tab:testbrowsers} lists all used browsers and under which platforms they were tested.

\begin{invisibletable}[Test browsers]{2}
  {p{0.25\textwidth} p{0.55\textwidth}}
  \label{tab:testbrowsers}%
  \ida{IE} 7 (Windows) & \idx{Firefox} 3.6 (Windows/Ubuntu)\\
  \ida{IE} 8 (Windows) & \idx{Opera} 10.5 (Windows)\\
  \idx{Safari} 4 (Windows) & \idx{Google Chrome} 5 (Windows/Ubuntu)\\
\end{invisibletable}

Initial testing also included \ida{IE} 6, but as the project advanced it was clear that support for this dated browser had to be dropped in order to maintain a sane development pace.
Even then, a considerable amount of the project time was spent fixing bugs with more modern versions of \ida{IE}, by far the most buggy of the five browsers.

The mobile version was properly tested just in the \idx{Safari} browser for the \idx{iPhone} during all development, and also in \idx{Google Chrome} under Ubuntu, given its a similar engine -- \idx{WebKit} -- and that it is much faster and easier to debug.
At the end, some further testing in the \idx{Android} emulator was done, but only to assess that the main interface was functional, knowing that the overall design was going to look different.

Continuously, and specially in every major step in the process, the application was tested in every browser.
Detailing here a formal collection of test cases will not add any useful information to this document, so that was dismissed.

Actually, real tests consisted on manually trying every path of the 24 use cases, the first 15 in the desktop browsers and the last 9 in the mobile browser.
Tables on \S\vref{ssub:usecasesold} and \S\vref{sub:new_use_cases} are exhaustive enough to describe the initial conditions and the expected outcome.
Also, in \S\vref{sec:interface_design} lots of pictures already depict the real outcome; they are not mockups but real screenshots of the application running under several browsers.
% section software_validation_and_verification (end)
\section{Interface Design} % (fold)
\label{sec:interface_design}

The interface design was not a straightforward task, but rather an iterative process where several alternatives were weighted.
Constant feedback from other members of the project helped shaping it until getting to the final design.
At the end of this work two interfaces were fully designed to bring the same functionality under both desktop and mobile constraints.

\subsection{Redesign} % (fold)
\label{sub:redesign}

The previous design had several flaws that needed to be fixed.
Also, since more functionality were implemented, additional abstractions needed to be designed.
Considering these two facts, the main points behind this redesign were:

\begin{description}
  \item[Organization] The previous design was quite chaotic, the elements were just scattered and the visual impact was odd.
  By constraining the design to a grid, elements fall into place, so the user impression is much better.
  \item[Simplicity] For example, the previous background was a little distracting, and the colors gave more importance to the background than to the foreground elements.
  By simplifying the shapes and desaturating the colors, now elements are more easily recognizable.
  
  Also, buttons are easier to distinguish now, because they are bright orange while the rest of the interface is desaturated.
  All new images also follow this convention, so that they do not feel out of place.
  Another example is the new trash icon: it is desaturated by default, but when the user drags something onto that icon, it activates showing the full color version of that image.
  \item[Increased contrast] Some texts in the previous design were very difficult to read: white over very light orange, for example.
  In this design this was a concern, and it results in a high contrast color palette: white over black, black over white.
  Other design tricks also help with the contrast, like adding shadows in some texts labels.
  \item[Smoothness] Animations have been heavily used in the interface, to give the interface a responsive feeling to the user actions.
  For example, when the user drags the icon it slightly fades and when the mouse is over a valid container it turns opaque.
  
  Also, when the icon is released onto an invalid container or when a session is duplicated/transferred, the icon smoothly \emph{flies} from one point to another.
  These nice little touches are not just superfluous gimmicks, but they improve the user experience.
  \item[Adaptability] The previous interface was static, all elements had fixed sizes and positions.
  Now, the elements are designed to be resized automatically to fit the available space.
  \item[Customization] The user can now move and resize his devices, so there are new elements to make this task easier.
  When the mouse is over a device, two handles appear: one to move it and another one to resize it.
  The user just has to drag them to see how the devices change live.
  
  Another visual element needed to be added for resizing the sidebar as a cue to the user, and it works mostly in the same way.
  The two panes of the interface resize accordingly with the position of that knob.
  \item[Fast recognizability] For the user it is more useful tagging the sessions with a still image from the video rather than a generic icon.
  That is the most effective way to visually discriminate two different sessions.
  \item[Condensed] On the one hand simplicity was one of our goals, but on the other hand more information will be added to the sidebar.
  This has been solved by adding \emph{tabs} to the sidebar so that the user can alternate between his buddy list and the new content list.
  
  Also, in the new content list an \emph{accordion} has been implemented to handle the different categories.
  So when content from one category is shown, the others are collapsed.
\end{description}

% subsection redesign (end)

\subsection{Mobile Design} % (fold)
\label{sub:mobile_design}

% subsection mobile_design (end)

% section interface_design (end)
\selectlanguage{spanish}
\begin{abstract}

Internet ha causado un tremendo impacto en muchos aspectos de nuestra vida cotidiana.
A medida que la sociedad se va acostumbrando a las facilidades de trabajar en línea, los hábitos cambian de manera acorde.
Aplicaciones que tradicionalmente se ejecutaban de manera nativa en la máquina del usuario se están, gradualmente, convirtiéndose en aplicaciones web ejecutadas remotamente.

Al mismo tiempo los navegadores han ido mejorando progresivamente hasta convertirse en potentes plataformas de desarrollo.
Esta mejora ha dado lugar a la aparición de aplicaciones web de una gran complejidad basadas en \idas{HTML}, \idas{CSS} y \idx{JavaScript}, distribuyendo una carga de procesamiento importante al cliente.
A la vez, se obtienen interfaces flexibles capaces de adaptarse a dispositivos muy dispares.

En este proyecto se documenta el desarrollo de una aplicación web avanzada cuyo propósito es controlar la reproducción de contenidos multimedia en varios dispositivos.
Esta aplicación se ha realizado en colaboración con \emph{Deutsche Telekom AG}, durante un estancia de seis meses en Berlín como parte del programa \emph{Erasmus Placement} en 2010.

Dicha aplicación se enmarca dentro del proyecto \idx{ScaleNet} (2005-2009), una Red de Siguiente Generación (\idas{NGN}) cuyo fin es un sistema que permita una integración escalable, rentable y eficiente de las diferentes tecnologías de acceso inalámbrico y por cable.
El componente desarrollado, la \emph{Interfaz de Administración de la Red Personal} (\idas{PNAI}), es solo una pequeña parte de \idx{ScaleNet} que sirve como ejemplo de aplicación sobre esta red.

Aunque la interfaz para estas operaciones ya existía, se solicitó un rediseño completo que integrara mayor funcionalidad y que ofreciera una experiencia de usuario más agradable.
Además de la interfaz principal para ordenadores de escritorio, también se explica el desarrollo de una interfaz web para dispositivos táctiles modernos.

\end{abstract}

\subsubsection{JavaScript Codebase} % (fold)
\label{ssub:jscodeold}

Finally, we get to the old \idx{JavaScript} codebase, the main place where the effort of this work was focused.
The code resides in the resources folder of the \ida{OSGi} bundle, where all the static public code is thrown: \idas{HTML}, \idas{CSS}, \idx{JavaScript}, images, etc.
When requested by a browser, these files are served by the \ida{OSGi}-based server, not \idx{Apache}.

\idx{JavaScript} classes are organized in several source files, each file acting as if they were traditional packages.
As discussed on \S~\ref{sec:javascript}, \emph{class} is not the right term to refer to a \idx{JavaScript} object, but it will be used through this document for simplicity's sake.

Additionally, in the old codebase the use of global variables is wildly used, and they are quite important to understand how the application works.
Since the \ida{UML} language is designed for \ida{OOP}, there is no easy solution to specify those variables in the same diagram.
However, a workaround is to directly copy how the \ida{DOM} works and put all those variables under the global \idc{window} object.

Figure~\ref{fig:class-pnai-js-old} shows a diagram with all the classes involved and the relationships between them.
Figure~\ref{fig:class-pnai-global-old} completes the picture adding the global object and the relationships between it and the custom classes.

\begin{figure}[htbp]
  \centering
    \includegraphics[width=\textwidth]{diagrams/class-pnai-js-old.1}
  \caption{Class diagram for the old \idx{PNAI}}
  \label{fig:class-pnai-js-old}
\end{figure}

\begin{figure}[htbp]
  \centering
    \includegraphics[width=\textwidth]{diagrams/class-pnai-global-old.1}
  \caption{The global \idx{JavaScript} object in the old \idx{PNAI}}
  \label{fig:class-pnai-global-old}
\end{figure}

The whole code revolves around two abstract classes: \idc{Container} and \idc{Session}.
Technically, they are not \emph{abstract} because the \idx{JavaScript} language does not offer this construction.
In reality, there are no objects created directly from these classes, but are created from subclasses that extend these classes.

A \idc{Container} is anything that can contain a session, or more specifically, any element where the user can drop a session.
All containers are stored in a \idc{Hash} (\idc{containerList}), where the key is the name of that container.
Each container have several slots where the sessions can go, this means that the container can hold up to that number of sessions at the same time.
The class provides methods to attach or detach session to those slots.

The \ida{DOM} representation for a container is a \idc{div} block.
A background image with the real appearance of the container is drawn inside of this \idc{div} in a \idc{img} element.
Each session the container \emph{owns} is drawn inside of this image, so the \idc{div}s for those sessions are children of the container's \idc{div}.

Each slot stores the coordinates where the session can be drawn, so they need to be calculated when the container's \idc{div} is created.
They also keep track of the session they belong to and the \ida{DOM} representation of that session (is the slot is not empty).

Every container can be enabled (online) or disabled (offline) at any time.
If it is disabled, it cannot have any session, and the background image changes to a more grayish image to reflect this new state to the user.

There are three types of containers: devices, buddies and the trash.
Since the number of devices is fixed in this interface for simplicity's sake, it is assumed that the user have a \ida{TV}, a \ida{NB} and a \ida{PDA}.

Each one is linked to a realistic image icon used by the interface.
The \ida{TV} and the \ida{NB} have four slots and the \ida{PDA} has only two slots, but there is no such constrain in the backend so this is only a cosmetic issue.

The trash is also considered a container, because it can \emph{receive} sessions.
To delete a session there is not button, the user has to explicitly drag the session and drop it in the trash.
Therefore, instead of storing them, the trash removes them from the interface.

A buddy is a special container, in that there can be more than one.
In the interface there is an icon for that buddy, but it behaves different from the devices.
That icon does not \emph{contain} the sessions, they are displayed at the right side of the buddy's name.
Each buddy can hold up to two sessions at the same time.
For the rest, it is pretty much the same as a generic container.

All the buddies are stored in a \idc{BuddyList} (\idc{buddyList}), apart from the rest of the containers.
The \idx{DOM} representation for this object is the sidebar with the buddies, another \idc{div} block.

A \idc{Session} in this page is simply a current session in the system.
Its \idx{DOM} representation it is also a \idc{div} with an image (a generic icon, not a thumbnail) and the name of the content playing.
All sessions are stored in a \idc{Hash} (\idc{sessionList}).
There are two kind of sessions: \idc{VideoSession} and \idc{AudioSession}, and they are mostly the same except for having different icons.

The \idc{moveto} method is the one in charge, not only of moving the session to a new container, but also of creating the visual representation for that session.
Unlike the containers, where the \idc{init} method creates the \idc{div}, sessions are not showed in the screen until the parent is explicitly set with this \idc{moveto} method.

Besides this custom code, there is one external dependency for handling drag\et{}drop in a convenient way.
The library is \idc{wz\_dragdrop}\footnote{Original site seems broken, but a copy is available on:\\ \url{http://gualtierozorni.altervista.org/dragdrop/dragdrop_e.htm}}, written by German author \emph{Walter Zorn}.
It comes from the early days of \idx{JavaScript}, trying to provide a cross-browser solution for this problem.
This library has been discontinued, but at the same time a lot of modern alternatives offer a better, simpler and cleaner approach.

To work with this library we have to explicitly set the items we want to be able to drag, in this case the sessions.
Then, function callbacks are available for several events, and we can redefine those functions to decide what to do next.

The \idc{my\_DropFunc} function is the one called by this library when the user drops the session into something (or, simply, stops dragging the session).
This is the only one that needed to be redefined, and basically it does this:

\begin{itemize}
  \item Get the id from the object that user has dropped.
  \item Get the session with that id (\idc{draggedSession}).
  \item Get the container where the session was before (\idc{sourceContainer}).
  \item Calculate the container where the session has been dropped using its coordinates and the function \idc{getContainer} (\idc{targetContainer}).
  \item If the target container is the trash, pass the action to the \idc{Java applet} using the \idc{notifyApplet} function.
  \item If there is a valid target container, show a popup menu giving the user the option to transfer the session, copy it or cancel the action.
  \item If there is a problem with the target or no target is chosen, cancel the action and move the session icon back to the original container.
\end{itemize}

Later, when the user selects an option, three scenarios can happen:

\begin{description}
  \item[Transfer session] The \idc{performHandover} function is called.
  In this function, the popup is closed and the \idx{Java applet} is notified using the \idc{notifyApplet} function.
  Until the backend answer, a panel with information (a \idc{div}) is shown using the \idc{showInfo} function, and the information about the action is stored in a \idc{PendingAction} object (\idc{pendingAction}).
  \item[Copy session] The \idc{performDuplication} function is called.
  This does the same as the previous case, but using another method name.
  \item[Cancel action] The \idc{cancelAction} function is called.
  Here, simply the popup menu is closed and the session icon is moved back to its original container.
\end{description}

After a while, the server will receive the request, process it and answer back to the \idx{Java applet}.
Then the \idx{Java applet} will trigger the callbacks accordingly to the action, to create/update/delete a session.
These callbacks (\idc{resultOK} and \idc{resultNotOK}) simply update the \ida{DOM} according to the new information and the pending action, attaching/detaching the session to/from the correct containers and hiding the info panel.

Additionally the \idc{newSession} function should be called when the user wants to duplicate a session.
From the point of view of the web interface, a duplicated session is completely unrelated to the original one, since the id is different.
So a full new session with new data and icon should be created.

There are other callbacks that the \idx{Java applet} may called at any time (and a lot at the load of the application): \idc{newSession}, \idc{deleteSession}, \idc{updateDevice}, \idc{deleteDevice}, \idc{updateBuddy} and \idc{deleteBuddy}.
Their names are very straightforward, and all of them update the stored data with the new information, changing the \ida{DOM} accordingly.
Since the devices in the screen are prefixed, the \idc{createDevice} and \idc{deleteDevice} functions act differently, only changing the online status of the device but never creating or deleting existing devices.

Finally, there is also some \idx{JavaScript} code to setup the page, analogous to the \emph{main} function in other languages like C or Java.
Listing~\ref{mainjs} shows this setup code.

\begin{lstlisting}[float=htbp,label=mainjs,language=javascript,caption=Setup code] % javascript
var myIMPI = 'hahn@imusu.mobile.dtrd.de';

var tv = new TV(40, 20, 380, 240);
tv.disable();
var laptop = new NB(40, 350, 380, 240);
laptop.disable();
var mda = new PDA(470, 20, 200, 300);
mda.disable();

var trash = new Trash(520, 430, 100, 150);
trash.enable();

buddyList.show(620, 20, 400, 600);
\end{lstlisting}

In this code, first of all the id of the user is set. This is a piece of code written dynamically on the fly by the Java server, and it is obviously different for every user.
Then the devices are created.
As stated before, it does not matter the real devices really owns the user, for this demonstrator it is assumed that the user has three devices.
By default the devices are disabled, because we do not know at the moment if they are online or not.

Then the trash and the buddy list (sidebar) are created.
For each of the previous elements, their coordinates and dimensions have been specified statically.
Those values were found by trial and error and hardcoded in the page.

% subsubsection jscodeold (end)
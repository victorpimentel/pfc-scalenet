\subsubsection{Components} % (fold)
\label{ssub:componentsold}

As explained in \S\vref{sub:demonstrator}, the software is mainly executed in three machines: two servers and a client.
Figure~\vref{fig:componentdiagramsold} shows all the relevant components running inside those machines and how they interact with each other.

\begin{figure}[htbp]
  \centering
    \includegraphics[width=\textwidth]{diagrams/component-pnai-old.1}
  \caption{Component diagram for the old \idx{PNAI}}
  \label{fig:componentdiagramsold}
\end{figure}

That component diagram brings some interesting aspects about the system, although we are only interested in several of them:

\begin{itemize}
  \item The \ida{SIP} protocol is used for the IMS part, but this is not important for the work done, as both the server and the client were not touched.
  \item The \ida{SSCON} manages the sessions and acts as the bridge between the actual services and the web interface shown to the user.
  Actions from the web interface are notified through \ida{UDP} calls.
  \item In this diagram, though, there are two parts missing that are almost completely irrelevant for this document.
  These are the application server component that streams the content and the multimedia player installed in the user device.
  The protocols used between these two components are not interesting for us, so for sake of simplicity, they don't appear here.
  The only important thing is that the \ida{SIP} client controls the external player, so the web interface does not handle the video streaming.
  \item There are two MySQL databases in use:
  \begin{description}
    \item[\idx{HSS DB}] This is the master database, and it is always up to date.
    It contains information about the user status, his buddy list and registered devices.
    \item[\idx{Personal DB}] This is an additional database that depends on the HSS DB.
    This database gather all the information needed for the \ida{PNAI}, since it contains all the relevant information from the \idx{HSS DB} plus the session information obtained from the \ida{SSCON}.
    Periodically, the \ida{SSCON} polls information from the master database and then updates this slave database, so the information can be a bit outdated compared to the \idx{HSS DB}.
    As stated in the diagram, the \ida{OSGi} component grabs the information it needs from this database, by polling it periodically.
  \end{description}
  \item Once the page is sent to the browser, the web interface continues talking with the server through a \ida{TCP} socket without reloading the page.
  This goes both ways, since it is used for sending actions and receiving data following a push model (so no delays polling).
  Since, at the time of developing the original application, there was no way to get that using only basic web standards, it uses a \idx{Java applet} to handle the socket communications.
\end{itemize}

\begin{figure}[htbp]
  \centering
    \includegraphics[width=\textwidth]{diagrams/sequence-handover-old.1}
  \caption{Sequence diagram for the old \idx{PNAI}}
  \label{fig:sequence-handover-old}
\end{figure}

Figure~\vref{fig:sequence-handover-old} shows the flow of the application in the scenario where the user wants to transfer a session from his device to one of his buddies.
Blue lines belong to the main logical flow, while the yellow ones belong to the video streaming process.
Other usage scenarios are very similar to this one, so they are ignored since this one explains well how and when they communicate.

As we can see, the web interface (written in \idx{JavaScript}) talks back and forth with the \idx{Java applet} using simple method calls, since all the code interface are directly available between them.
Then the \idx{Java applet} translates those calls to strings with the method names and parameters and sends it to the \ida{OSGi} using the \ida{TCP} connection.

At the right part of the diagram, it is clear that the sessions are controlled using \ida{SIP} messages.
Given that \idx{ScaleNet} is a very decentralized network by design, the \ida{SSCON} delegates to the \ida{SIP} clients running in the devices all the talking with the \idx{Application Server}.

It is interesting to note that everything is mostly asynchronous, so there is not a lot of calls that block.
Therefore the use of threads and callbacks is widespread in all components.

% subsubsection componentsold (end)
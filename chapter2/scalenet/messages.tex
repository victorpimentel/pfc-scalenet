\subsubsection{Messages} % (fold)
\label{ssub:messagesold}

Of all the messages sent inside the system, the most important ones for us are sent though the \ida{TCP} socket.
These are processed and generated by the web interface and the \ida{OSGi} backend, but they follow a different format depending which component sends the message.

Messages generated by the backend consist of serialized objects following a very simple format.
There are three kind of data objects that can be sent over the wire: devices, buddies and sessions.

A serialized object is a string that starts with the type of the object (\idc{device}, \idc{buddy}, \idc{session}), followed by the vertical bar character `|' as delimiter.
Then the attributes for that object are appended one by one, separated by the same delimiter.
If the values are not strings, they are converted directly, for example a boolean with value \texttt{true} will be passed as the string \texttt{"true"}.

The client does not know when a transfer or duplication happens, it is only notified of creation and deletion of things.
When a status update happens, such as a device going online, it is notified as the creation of an object.
Therefore the client must keep track of the objects received and realize that it is an update of a previously created object.

For example, as seen in Figure~\vref{fig:sequence-handover-old}, when a transfer happens the interface received two commands, first creating a new session and then deleting the original session.

To notify that a new object needs to be created in the view, the backend just sends the serialized object, without adding anything else.
To notify that an existing object has to be deleted from the view, the string is the same but preceded by the text \texttt{"deleted|"} (that is, \emph{deleted} and the delimiter).

Table~\vref{tab:notificationsexamples} comprises all the different messages that can be sent from the \ida{OSGi} backend to the \idx{Java applet} with examples.

\begin{generictable}[Format of the notifications sent to the applet]{2}
  {|p{0.21\textwidth}|p{0.69\textwidth}|}
  {\generictitletwo{Notification}{Format \et{} Example}}
  \label{tab:notificationsexamples}%
  Create/update device & \texttt{device|\emph{impi}|\emph{impu}|\emph{online}} \newline
    \texttt{device|hahn@imusu.mobile.dtrd.de\newline
    $\hookrightarrow$|tv.hahn@imusu.mobile.drtd.de|true} \\ \hline
  Delete device & \texttt{deleted|device|\emph{impi}|\emph{impu}|%
    \emph{online}} \newline
    \texttt{deleted|device|hahn@imusu.mobile.dtrd.de\newline
    $\hookrightarrow$|tv.hahn@imusu.mobile.drtd.de|false} \\ \hline
  Create/update buddy & \texttt{buddy|\emph{id}|\emph{name}|\emph{online}}%
    \newline
    \texttt{buddy|3|hahn@imusu.mobile.dtrd.de|false} \\ \hline
  Delete buddy & \texttt{deleted|buddy|\emph{id}|\emph{name}|\emph{online}}%
    \newline
    \texttt{deleted|buddy|3|hahn@imusu.mobile.dtrd.de\newline
    $\hookrightarrow$|false} \\ \hline
  Create/update session & \texttt{session|\emph{id}|\emph{type}|\emph{name}%
    |\emph{owner}|\emph{initiator}|\emph{impi}\newline
    $\hookrightarrow$|\emph{impu}|\emph{icon}}
    \newline
    \texttt{session|3|video|NASA\newline
    $\hookrightarrow$|hahn@imusu.mobile.dtrd.de\newline
    $\hookrightarrow$|hahn@imusu.mobile.dtrd.de\newline
    $\hookrightarrow$|hahn@imusu.mobile.dtrd.de\newline
    $\hookrightarrow$|laptop.hahn@imusu.mobile.dtrd.de\newline
    $\hookrightarrow$|http://imusu.mobile.dtrd.de/img/icon.png} \\ \hline
  Delete session & \texttt{deleted|session|\emph{id}|\emph{type}|\emph{name}%
    |\emph{owner}\newline
    $\hookrightarrow$|\emph{initiator}|\emph{impi}|\emph{impu}|\emph{icon}}
    \newline
    \texttt{deleted|session|3|video|NASA\newline
    $\hookrightarrow$|hahn@imusu.mobile.dtrd.de\newline
    $\hookrightarrow$|hahn@imusu.mobile.dtrd.de\newline
    $\hookrightarrow$|hahn@imusu.mobile.dtrd.de\newline
    $\hookrightarrow$|laptop.hahn@imusu.mobile.dtrd.de\newline
    $\hookrightarrow$|http://imusu.mobile.dtrd.de/img/icon.png} \\ \hline
\end{generictable}

Going back to Figure~\vref{fig:sequence-handover-old}, the message sent in \texttt{TCP(2)} was a \emph{Create/Update session} notification, while the message sent in \texttt{TCP(2)} was a \emph{Delete session} notification.
In both cases, the \idx{Java applet} just parsed the messages and passed the arguments to the right \idx{JavaScript} callbacks.

On the other hand, messages sent by the \idx{Java applet} are requests from the user, for actions that he wants to complete relating sessions.
These actions can be the deletion of a session, its transfer or its duplication.

It does not matter which kind of origin (device or buddy) or destination it goes, because dealing with unique \ida{SIP} identifiers (\idc{impi}) makes sure that every element is treated equally.

They follow a query string format, which is the usual way to pass data as part of a \ida{URL}.
This string is passed directly to the \ida{TCP} socket, without any additional encoding.

Table~\vref{tab:requestsexamples} lists the different messages that can be sent from the \idx{Java applet} to the \ida{OSGi} backend with examples.

\begin{generictable}[Format of the requests sent from the applet]{2}
  {|p{0.21\textwidth}|p{0.69\textwidth}|}
  {\generictitletwo{Request}{Format \et{} Example}}
  \label{tab:requestsexamples}%
  Copy session & \texttt{event=duplicate\&uid=\emph{uid}\&%
    source=\emph{source}\newline
    $\hookrightarrow$\&sid=\emph{sid}\&destination=\emph{target}}\newline
    \texttt{event=duplicate\newline
    $\hookrightarrow$\&uid=hahn@imusu.mobile.dtrd.de\newline
    $\hookrightarrow$\&source=laptop.hahn@imusu.mobile.drtd.de\newline
    $\hookrightarrow$\&sid=458215\newline
    $\hookrightarrow$\&destination=steffen@imusu.mobile.drtd.de} \\ \hline
  Transfer session & \texttt{event=handover\&uid=\emph{uid}\&%
    source=\emph{source}\newline
    $\hookrightarrow$\&sid=\emph{sid}\&destination=\emph{target}}\newline
    \texttt{event=handover\newline
    $\hookrightarrow$\&uid=hahn@imusu.mobile.dtrd.de\newline
    $\hookrightarrow$\&source=laptop.hahn@imusu.mobile.drtd.de\newline
    $\hookrightarrow$\&sid=458215\newline
    $\hookrightarrow$\&destination=tv.hahn@imusu.mobile.drtd.de} \\ \hline
  Stop session & \texttt{event=delete\&uid=\emph{uid}\&%
    source=\emph{source}\&sid=\emph{sid}}\newline
    \texttt{event=delete\&uid=hahn@imusu.mobile.dtrd.de\newline
    $\hookrightarrow$\&source=laptop.hahn@imusu.mobile.drtd.de\newline
    $\hookrightarrow$\&sid=458215} \\ \hline
\end{generictable}

\texttt{TCP(1)} in Figure~\vref{fig:sequence-handover-old} is a typical \emph{Transfer session} request, generated by a \idx{Java applet} after the \idx{JavaScript} requested it upon a user's action.
Additionally, \texttt{UDP(1)} follows the same format and \texttt{SIP(1)} contains the same information but in \ida{SIP} format.
Other messages from that figure are not explained with more detail because the pieces of code that deal with MySQL or SIP are in other modules apart from the web application.

% subsubsection messagesold (end)
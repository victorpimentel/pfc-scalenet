\section{JavaScript Framework: \idx{MooTools}} % (fold)
\label{sec:mootools}

Before explaining why \idx{MooTools} has been chosen for this application, another important question needs to be resolved.

\subsection{Why Use a \idx{JavaScript} Framework?} % (fold)
\label{sub:whymootools}

There are several reasons that lead to this conclusion, the most important are:

\begin{itemize}

  \item Because we want to support different browsers.
  If we do not use a framework a lot of time would be spent debugging the huge differences between \idx{Internet Explorer} and the rest of the browsers.

  \item Because we want to facilitate the development, since usually these frameworks cover several holes in the \idx{JavaScript} specification that allows us fixing common issues with less code.

  \item Because we want the interface to have advanced effects.
  We could just search for several scripts that makes one individual effect, but that will result in redundancies, differences in quality code and waste time in searching.

\end{itemize}

% subsection whymootools (end)

\subsection{Making the Decision} % (fold)
\label{sub:decision}

By the previous standards, we have plenty of options to choose from:
\idx{jQuery}\footnote{\url{http://jquery.com/}},
\idx{Prototype}\footnote{\url{http://www.prototypejs.org/}},
\idx{Dojo}\footnote{\url{http://www.dojotoolkit.org/}},
\idx{YUI}\footnote{\url{http://developer.yahoo.com/yui/}},
\idx{GWT}\footnote{\url{http://code.google.com/webtoolkit/}},
\idx{Ext JS}\footnote{\url{http://www.extjs.com/}}, etc.
Overall, these are very popular and they offer high quality and plenty of functionality.
However, for this particular project, and after some consideration, \idx{MooTools}\footnote{\url{http://mootools.net}} was considered the best option.
The reasons for this decision are:

\begin{description}

  \item[Compact] It has a low footprint on the site load because it is reasonably lightweight for the functionality it offers.
  Particularly, it is more optimized in this aspect than \idx{Prototype}, \idx{YUI} or \idx{Dojo}, but it is also slightly more compact than \idx{jQuery}.

  \item[Modular-Based] Because of that, the installation can be customized to get only the modules we need, and the creation of our own extensions is easier.

  \item[Compatible] It has been tested with most browsers: \idx{Internet Explorer} 6+, \idx{Firefox} 2+, \idx{Opera} 9+, \idx{Safari} 2+ (and other \idx{Webkit}-based browsers, like \idx{Chrome}).

  \item[Functional] It offers all the functionality required for the first phase of the project: \idx{drag\et drop}, resize, animations, etc.

It also offers other functionality like \idas{AJAX} support, \idc{Hash} handling or \idc{Cookie} handling, that ease the development in different browsers.

  \item[Object-Oriented] By adding \emph{Classes} to \idx{JavaScript}, an abstraction that it is perfect for this application, since the server code is written is \idx{Java}.

This way, we can use similar concepts both in the server and in the client. Moreover, the inherited code for \idx{ScaleNet} already used \idx{JavaScript} objects.

  \item[Extensive] It also has a repository for official plugins called  \idx{MooTools More} (with similar code quality and documentation to the \idx{MooTools Core}) and other third-party plugins can be found in the web.

  \item[Well-documented] It has extensive documentation for every class of the  framework.

  \item[Well-structured] Its structure is perfect for a professional web application.
  Frameworks like \idx{jQuery} are more focused in reducing the  lines of code that in encouraging robust coding.
  \idx{MooTools} also helps reducing the lines of code, but it has more tools for writing code in a very modular, reusable and robust way, for example by using classes and other abstractions.

  It also improves the readability of the code, something hard to do in \idx{JavaScript}.
  Another important point of this framework is that it is based on prototype extensions (mainly \idc{DOM} extensions), so the syntax is very Object-Oriented and the code seems very clean.

  \item[Used by the \ac{APE} server\index{APE server}]
  So if we use that component, it will be very straightforward to write extensions in \idx{JavaScript} also in the server.
  This will mean that we could use the same coding style and the same tools in the server as in the client.

\end{description}

% subsection decision (end)

% section mootools (end)

\subsection{History} % (fold)
\label{sub:jshistory}

\emph{Brendan Eich} originally developed the \idx{JavaScript} language working at \emph{Netscape Corp.} under the name of \emph{Mocha}, renamed to \emph{LiveScript} and then again to finally \idx{JavaScript}.

First implemented in \idx{Netscape} Navigator 2.0 in 1995, and contrary to whatever the name may lead to think, it has little to none to do with \idx{Java}.
Indeed, the name only served more marketing purposes.
The competitor, Microsoft, added support for the essentially the same language in \ida{IE} 3 the next year, but called \emph{JScript} to avoid trademark issues.

\begin{wrapfigure}{r}{0.5\textwidth}
  \centering
    \includegraphics[width=0.48\textwidth]{logo-javascript}
  \caption[JavaScript logo]{Since there is no official \idx{JavaScript} logo, here is a rhino}
  \label{fig:javascript-logo}
\end{wrapfigure}

Then \idx{Netscape} delivered the language to the \ida{ECMA} for standardization.
The effort culminated in 1997 with the first edition of a new open standard called \emph{ECMAScript}, named as a compromise in the dispute between Netscape and Microsoft.
For some reason, that name was never popularized, instead the original term \idx{JavaScript} is used in almost every situation.

Being closely linked with a markup language that even non-programmers and \emph{amateurs} could code, \idx{JavaScript} was initially disregarded as a toy by many traditional developers.
Other apparent limitations of the language, a very lenient parser, performance issues and compatibility problems between browsers did not help its popularity either.

However, the advent of \ida{AJAX}, its ubiquity in browsers, better/faster parsers and the need for more complex web interfaces put \idx{JavaScript} back in every professional web developer toolbox.
Since then a multitude of framework and libraries have been released to ease the development, with its usage growing broadly not only in browsers but even in server applications.

% subsection jshistory (end)
\subsection{Quick Overview of the Language} % (fold)
\label{sub:overviewjavascript}

At first glance the syntax may seem a simplified version of \idx{Java} (or other C-based language), but much more direct (no classes, packages, etc.).
Make no mistake, under the hood \idx{JavaScript} hides a lot of powerful constructions that makes it flexible enough to suit diametrical programming paradigms.
Some of the most important features are:

\begin{description}
  \item[Scripted] The browser receives the plain source code, and then it is interpreted and run.
  The big advantage is that it does not needed to be compiled beforehand, the traditional disadvantage is that it is slower than native code or even \idx{Java} byte code.
  
  Nevertheless, a lot of improvements have been deployed lately in browser engines, and new techniques like \ida{JIT} nearly close that gap.
  On the other hand, similarly to other script-based languages, it is easy to end up with a low-quality codebase, since inexperienced programmers can quickly write working code.
  \item[Imperative statements] It supports the basic imperative paradigm with the typical if statements, while and for loops, global functions, etc.
  It is not required to think about objects to make a script, with those things it can be possible to write a full-fledged application if wanted.
  \item[Object based] Internally, in \idx{JavaScript} everything is an object, not only variables but even functions or undefined values.
  However, there are no classes like in \idx{Java}, to create a new object it can be specified in a literal notation or using a constructor (a plain function).
  \item[Prototype based] The language is classless, but inheritance between objects is possible through the use of \emph{prototypes}.
  A prototype is just an object used as a template from which to get the initial properties for a new object.
  If an object is asked for a property (also known as \emph{attributes} and \emph{methods} in \idx{Java} jargon) it does not have, its parent is asked, and then again until the property is found or until the root object is found (the end of the \emph{chain}).
  Listings~\vref{oopjs} and \vref{oopjava} show how inheritance works in \idx{JavaScript} compared to \idx{Java}.
  
\begin{lstlisting}[float=htbp,label=oopjs,language=javascript,caption=Inheritance in JavaScript]
  function Employee() {
    this.name = "";
    this.dept = "general";
    this.hello = function() {
      console.log("Hello, I'm " + this.name);
    };
  }
  function Engineer() {
    this.dept = "engineering";
    this.projects = [];
  }
  Engineer.prototype = new Employee;

  var john = new Engineer();
  john.name = "John Doe";
  john.hello();
\end{lstlisting}

\begin{lstlisting}[float=htbp,label=oopjava,language=java,caption=Inheritance in Java]
  public class Employee {
    public String name;
    public String dept;
    public Employee() {
      this.name = "";
      this.dept = "general";
    }
    public void hello() {
      System.out.println("Hello, I'm " + this.name);
    }
  }
  public class Engineer extends Employee {
    public String[] projects;
    public Engineer() {
      this.dept = "engineering";
      this.projects = new String[0];
    }
  }
  Employee john = new Engineer();
  john.name = "John Doe";
  john.hello();
\end{lstlisting}
  
  It is easy to demonstrate that this abstraction can be more powerful than classes because a similar class-based \ida{OOP} behavior can be achieved with prototypes but the opposite is not true.
  \item[First-class functions] As said before, \idx{JavaScript} functions are objects, so they have properties and methods and can be assigned to variables as every other object.
  Moreover, these functions can be passed as arguments, returned as other function values and invoked in any moment and scope using multiples ways (such as the \verb+()+ operator).
  
  Functions do not need to be defined in the global scope, they can be defined inside other functions.
  Nowadays the use of closures (specifically anonymous functions) is considered a very good practice to avoid cluttering the global scope with variables.
\end{description}

As stated repeatedly, every variable in \idx{JavaScript} contains a reference to an object.
However, that does not mean that values do not have a type, and there are several primitive types available in the language.
In any case, even literals of these types have accessible properties like any other object.

\begin{description}
  \item[Undefined and Null] Before assigning some value, a variable holds the value \idc{undefined}.
  This \idc{undefined} is the only object of, pun intended, type \idc{undefined}, a full-fledged object that can be even used in assignments.
  On the other hand, \idc{null} is never used by \idx{JavaScript}, so if some variable holds this value, it must have been set programmatically by the developer.
  Although the type \idc{null} is apparently \idx{Object}, internally it is also the only value of type \idc{null}.
  \item[Number] In \idx{JavaScript} there is no external distinction between integers, floating point values and others, every literal number is treated equally and has the same properties.
  Internally they are all doubles, floating points numbers with an accuracy nearly 16 significant digits, so some real numbers cannot be represented.
  \item[String] Strings are delimited by double quotes and single quotes indistinctly, the only difference is which quotes need to be escaped inside of the string.
  \item[Boolean] As usual, there are only two values possible in this data type: \texttt{true} or \texttt{false}.
\end{description}

For all of these types (except undefined/null), there are object constructors, but the preferred way to initialize them is through plain literals.
Besides these primitives data types, there are also several native objects built in the language:

\begin{description}
  \item[Array] As other C-based languages, arrays are available to store a collection of values in a variable using integers as keys.
  To create an empty literal array (also preferred to the Array constructors), two square brackets can be used.
  This bracket notation can also used for accessing object properties, so it can be confused with associative arrays while the language does not support this directly.
  \item[Date] A simple object to store and manipulate dates down to a milliseconds.
  \item[Error] Similarly to exceptions in other languages, this kind of objects can be thrown in case of error (using the throw clause) and caught for error handling (using the try/catch block).
  \item[Math] This object cannot be \emph{instantiated} and it only contains math-related constants and functions.
  \item[Regular Expression] Literals between backslashes represents regular expressions, and they hold some useful methods to deal with search and replace in strings.
  The \texttt{RegExp} constructor can be used but, again, it is not recommended.
  \item[Function] Functions are declared by using the keyword \texttt{function}, an optional name, the arguments and the body block.
  The object constructor, though also available, is very ill-advised, as it relies on strings.
  \item[Object] The rest of the values are of type \idx{Object} or indirectly inherit from this type.
  Objects literals are represented by curly brackets and a comma-separated list of keys/values separated by colons.
  This literal syntax is the base for \ida{JSON}, a notation used for data exchange that can be natively used in \idx{JavaScript}.
\end{description}

% subsection overviewjavascript (end)
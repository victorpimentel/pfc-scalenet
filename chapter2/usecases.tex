\subsubsection{Use Cases} % (fold)
\label{ssub:usecasesold}

\begin{center}
  \begin{usecase}[Stop a session of a device]
    \label{tab:usecasestopdevice}%
    \usecaseactor{System user}
    \usecasepre{A session is already running on a device, and it is showing in the \ida{PNAI} interface inside of that device.}
    \usecasepost{Session must terminate, i.e., the content must stop playing. The user must be notified with a popup and the session icon must be deleted from the \ida{PNAI} interface.}
    \usecasemain{
      \begin{usecasepath}
        \item User starts dragging the session icon.
        \item A copy of the session icon appears under the user's cursor, and follows the cursor until the user drops it.
        \item User drops the cloned session icon into the trash.
        \item A popup appears to notify the user that the action is in progress and the cloned session icon is deleted from the view.
        \item The content stops playing.
        \item The popup disappears and the original session icon is deleted from the view.
      \end{usecasepath}
    }
    \usecasealt{1}{
      \begin{usecasepath}[b]
        \setcounter{enumi}{2}
        \item User drops the session into a blank space.
        \item Action is cancelled.
      \end{usecasepath}
    }
    \usecasealt{2}{
      \begin{usecasepath}[c]
        \setcounter{enumi}{4}
        \item There is an error with the server and the content keeps playing.
        \item The content of the popup changes to notify the user that there was an error with the server and the action could not be completed.
        After 5 seconds it disappears.
        \item Action is cancelled.
      \end{usecasepath}
    }
    \usecasealt{3}{
      \begin{usecasepath}[d]
        \item There is an error with the server and the content keeps playing.
        \item The content of the popup changes to notify the user that there was an error with the server and the action could not be completed.
        After 5 seconds it disappears.
        \item Action is cancelled.
      \end{usecasepath}
    }
  \end{usecase}
\end{center}

The use case for terminating a session that a buddy is playing and that we own is very similar:

\begin{center}
  \begin{usecase}[Stop a session of a buddy]
    \label{tab:usecasestopbuddy}%
    \usecaseactor{System user}
    \usecasepre{A session owned by the user is running on a device, and it is showing in the \ida{PNAI} interface near that buddy's name.}
    \usecasepost{Session must terminate, i.e., the content must stop playing. The user must be notified with a popup and the session icon must be deleted from the \ida{PNAI} interface. The buddy is \emph{not} notified, the content stops without warning.}
    \usecasemain{
      \begin{usecasepath}
        \item User starts dragging the session icon.
        \item A copy of the session icon appears under the user's cursor, and follows the cursor until the user drops it.
        \item User drops the cloned session icon into the trash.
        \item A popup appears to notify the user that the action is in progress and the cloned session icon is deleted from the view.
        \item The content stops playing.
        \item The popup disappears and the original session icon is deleted from the view.
      \end{usecasepath}
    }
    \usecasealt{1}{
      \begin{usecasepath}[b]
        \setcounter{enumi}{2}
        \item User drops the session into a blank space.
        \item Action is cancelled.
      \end{usecasepath}
    }
    \usecasealt{2}{
      \begin{usecasepath}[c]
        \setcounter{enumi}{4}
        \item There is an error with the server and the content keeps playing.
        \item The content of the popup changes to notify the user that there was an error with the server and the action could not be completed.
        After 5 seconds it disappears.
        \item Action is cancelled.
      \end{usecasepath}
    }
  \end{usecase}
\end{center}

% subsubsection usecasesold (end)

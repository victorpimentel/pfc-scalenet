\section{Server Programming Language: \idx{PHP}} % (fold)
\label{sec:php}

\ida{PHP}\footnote{\url{http://www.php.net/}} is one of the languages used to build web applications in \idx{ScaleNet}, and it is one of the most popular languages for building web applications in general.
Not only most of the portal is written in this language but, eventually, the \ida{PNAI} interface will be ported from an \ida{OSGi} bundle to a \ida{PHP} script.

\subsection{History} % (fold)
\label{sub:phphistory}

Originally, \ida{PHP} was created in 1995 by \emph{Rasmus Lerdorf} as a set of \ida{CGI} scripts written in \idx{C} that parsed \ida{HTML} files.
The goal was being able to call specific \idx{C} routines and show its output when a page was visited, by directly embedding the code in the \ida{HTML} source.
In the next years this open source project became a full-fledged parser, creating a new generic language.

\begin{wrapfigure}{r}{0.5\textwidth}
  \centering
    \includegraphics[width=0.48\textwidth]{logo-php}
  \caption{\idx{PHP} logo}
  \label{fig:logo-php}
\end{wrapfigure}

In 1997 \emph{Zeed Suraski} and \emph{Andi Gutmans} rewrote that parser to include more functionality and released it as \ida{PHP} 3.
Then, specially after \ida{PHP} 4, the language started to be widely used.
\ida{PHP} has gained a lot of popularity for web development projects and it is used in big websites like Yahoo, Facebook and Wikipedia.

In 2004 \ida{PHP} 5 was released, adding \ida{OOP} capabilities to the language.
However, it is compatible with \ida{PHP} 4 scripts, so it is optional to work in an Object-Oriented way.
This is the most recent version and it is the one installed in the \idx{IMS} demonstrator.

% subsection phphistory (end)

\subsection{Quick Overview of the Language} % (fold)
\label{sub:overviewphp}

A \ida{PHP} file is basically an \ida{HTML} file with the \texttt{.php} extension that it is usually stored in the public folder of the server (\idx{Apache}, \ida{IIS} or other supported server).
When that page is requested, the server calls the \ida{PHP} module, that parses that file and returns a normal \ida{HTML} page.
Therefore it is a interpreted language, so no compilation is needed.

The interesting things happen within its delimiters
(usually <?php and ?>), where the \ida{PHP} code is written.
Outside of these delimiters the text is treated as a normal \ida{HTML} soup and therefore not processed.
Listing~\vref{phpexample} shows an example of \ida{PHP} code embedded within \ida{HTML} code and Listing~\vref{phpexampleafter} shows the resulting \ida{HTML} output.

\begin{lstlisting}[float=htbp,label=phpexample,language={[phpoo]php},alsolanguage=html,caption=\idx{PHP} code embedded within \idx{HTML} code] % java
  <!DOCTYPE html>
  <html>
    <head>
      <title>PHP Test</title>
    </head>
    <body>
    <?php
    for ($i = 0; $i < 5; $i++) {
      echo "<p>Hello World " . $i . "</p>\n";
    }
    ?>
    </body>
  </html>
\end{lstlisting}

\begin{lstlisting}[float=htbp,label=phpexampleafter,language=html,caption=Resulting \idx{HTML} code]
  <!DOCTYPE html>
  <html>
    <head>
      <title>PHP Test</title>
    </head>
    <body>
    <p>Hello World 0</p>
    <p>Hello World 1</p>
    <p>Hello World 2</p>
    <p>Hello World 3</p>
    <p>Hello World 4</p>
    </body>
  </html>
\end{lstlisting}

Its syntax is very similar to \idx{C}, with equivalent constructions (if conditions, for and while loop, functions, etc).
Some notable exceptions are that variables must start with a dollar sign character (\texttt{\$}), and that a dot is used for concatenating strings instead of the most traditional plus sign.

Variables are dynamically typed, so the programmer does not need to specify types. A variable's type is determined by the context in which that variable is used, so any variable can hold different types during an execution.

One of the strengths of \ida{PHP} is the wide range of utility functions bundled in the processor and in additional extensions usually included in distributions.
Although objects are supported in \ida{PHP} 5, it is not discussed here because the scripts in \idx{ScaleNet} do not use them.

Global variables like \idc{\$\_GET}, \idc{\$\_POST} or \idc{\$\_SERVER} offer access to information sent from the browser, so it is commonly used to pass parameters to a \ida{PHP} script (through the \ida{URL} or forms in the page).
On the other hand \idc{\$\_SESSION} and \idc{\$\_COOKIE} allow to store some data through the same \emph{visit}, even across different scripts.

Since the main goal of the language is generating \ida{HTML} content, the most common functions are ones that \emph{print} text in the output, like \idc{echo}.
Because \idx{MySQL} is quite popular in web development, there is an extension with functions to interact with those databases.

Other commons functions are \idc{include} (to, ehem, include other \ida{PHP} source files), \idc{isset} (to know is a variable is set), \idc{die} (to terminate the execution of the script at any moment), \idc{header} (to set \ida{HTTP} headers) and diverse string/array manipulation functions.

% subsection overviewphp (end)

% section php (end)